\chapter{Einleitung}
\label{einleitung}
Die \ac{KI} hat in den letzten Jahren immer mehr an Bedeutung gewonnen und wird auch in Zukunft eine wichtige Rolle spielen. Sie wird in den unterschiedlichsten Lebensbereichen eingesetzt. Von personalisierten Produktempfehlungen in Online-Shops über autonome Fahrzeuge bis hin zur Diagnose von Krankheiten. In Arizona gibt es zum Beispiel einen autonomen Taxi-Service, der ohne einen Menschen hinter dem Lenkrad fährt. Es wird sogar gesagt, dass die Fahrweise der \ac{KI} sicherer sei als die eines Menschen. \cite{gibbs2017google}

Dies wirft bei einigen Menschen Bedenken auf, da Sie fürchten, dass die \ac{KI} den Menschen in verschiedenen Bereichen wie auch den traditionellen Taxifahrer überflüssig machen. Federspiel hingegen betont in seinem Artikel den Nutzen der Ersetzung von repetitiver Arbeit, da sie dem Menschen dabei hilft, sich auf andere Bereiche zu fokussieren. \cite{Federspiele010435}

Im Bildungsbereich bietet die \ac{KI} erhebliche Potenziale. Sie kann Lehrkräfte unterstützen, indem sie individuelles Feedback für Schülerinnen und Schüler generiert und maßgeschneiderte Lernpläne erstellt. Diese können Lehrern als wertvolle Empfehlungen für den Unterricht dienen und den Schülern helfen, gezielt an ihren Schwächen zu arbeiten. \cite[S.3 ff.]{fraiwan2023reviewchatgptapplicationseducation} 

Trotzdem ist es wichtig, \ac{KI} immer als Hilfsmittel zu betrachten. Die Ergebnisse müssen sorgfältig geprüft werden, da sie Fehlinformationen enthalten können und die Antworten nicht immer konsistent sind. Der Einsatz von \ac{KI}, speziell von Sprachmodellen und Chatbots, kann zu einer potenziellen Abhängigkeit führen. Es besteht die Möglichkeit, dass Forscher, Lehrer und Studenten ihre eigenen kritischen Denkfähigkeiten vernachlässigen. Dadurch können in ihrer Arbeit Fehler und Ungenauigkeiten auftreten. \cite[S.10 ff.]{fraiwan2023reviewchatgptapplicationseducation}

Darüber hinaus besteht das Problem des sogenannten „Bias“ in \ac{KI}-Systemen. Ein Bias ist eine fehlerhafte Neigung zu bestimmten Faktoren in einer Studie. Da die von Chatbots generierten Antworten auf den ihnen zur Verfügung stehenden Datensätzen basieren, können in diesen Datensätzen Fehler, Verzerrungen oder Biases auftreten. Dies führt dazu, dass die Ergebnisse nicht der Realität entsprechen \cite[S.3 ff.]{anslinger2021faire}. Ob diese Bias-Verzerrungen auch auftreten, wenn eigene Textdatensätze wie zum Beispiel eine Bachelorarbeit vorgegeben werden, ist noch offen. Es stellt sich auch die Frage, ob die \ac{KI}, ähnlich wie beim autonomen Fahren, bessere Ergebnisse bei der Textzusammenfassung liefern kann als ein Mensch.

\section{Motivation}
Die zunehmende Verbreitung von \ac{KI} in verschiedenen Branchen und Anwendungsbereichen weckt nicht nur Interesse, sondern macht es auch notwendig, ihre Möglichkeiten voll auszuschöpfen. Unternehmen, Forschungsinstitute und Bildungseinrichtungen müssen sich mit der Verarbeitung von riesigen Datenmengen auseinandersetzen. Hier kann \ac{KI} helfen, Daten zu analysieren, Muster zu erkennen und Vorhersagen zu treffen. Um diese Potenziale optimal nutzen zu können, müssen jedoch zunächst Systeme entwickelt werden, die auf die jeweiligen Anforderungen zugeschnitten sind.

Ein besonders relevantes Anwendungsfeld für \ac{KI} ist die Automatisierung repetitiver und zeitintensiver Aufgaben. In der Wissenschaft kann es aufwändig sein, Texte zu erstellen und zu bearbeiten. Forscher verbringen viel Zeit damit, Texte zu schreiben, zu formatieren und zu lesen. Diese Zeit ist besser investiert, wenn man sie für kreative und innovative Aufgaben nutzt. Es wäre daher sinnvoll, eine Anwendung zu entwickeln, die derartige wiederkehrende Aufgaben übernimmt.

Darüber hinaus können vorhandene Erfahrungen in der Entwicklung von Webanwendungen genutzt und erweitert werden. Vorerfahrungen aus früheren Projekten wie \textit{BattleSprout}, bieten eine Basis \cite{battlesprout}. Das dort erworbene Wissen über die Gestaltung von Webanwendungen kann direkt in die Entwicklung neuer Webanwendungen einfließen.

\section{Ziel der Arbeit}
Ziel dieser Arbeit ist die Entwicklung eines webbasierten Bericht Generators namens \textit{Reforge}. Dieser soll in der Lage sein, eingereichte Abschlussarbeiten zusammenzufassen und in spezifischen Formaten auszugeben. Die entwickelte Anwendung soll zudem den Studierenden und Dozenten dabei helfen, den Vorgang der Erstellung von technischen Berichten zu automatisieren und zu vereinfachen. Darüber hinaus wird bei den Endprodukten geprüft, inwieweit die originalen Input-Dateien im Vergleich zu den generierten Berichten verändert wurden, um die Genauigkeit der \ac{KI}-generierten Inhalte zu überprüfen.

Im Rahmen der Bachelorarbeit wird untersucht, welche Erkenntnisse aus der Entwicklung und dem Einsatz dieser Anwendung gewonnen werden können und ob der Einsatz eines solchen Generators sinnvoll ist. Der Fokus liegt dabei auf der Qualität der erzeugten Berichte und der Konzeption der Anwendung.

Folgende Fragen sollen beantwortet werden:
\label{kernfragen}
\begin{itemize}
    \item Welche Architektur ist geeignet für die Entwicklung einer Webanwendung, die die automatische Erstellung von TechReports unterstützt?
    \item Wie kann ein \ac{LLM} zur automatischen Zusammenfassung in die Webanwendung integriert werden?
    \item Wie lässt sich die Qualität der automatisch generierten Berichte im Vergleich zu den manuell erstellten Versionen bewerten?
\end{itemize}

\section{Aufbau der Arbeit}

Diese Arbeit gliedert sich in mehrere Kapitel, die von den theoretischen Grundlagen über den aktuellen Forschungsstand bis hin zur Implementierung und Evaluierung des Projekts \textit{Reforge} führen. Kapitel zwei behandelt die notwendigen theoretischen Grundlagen, die für das Verständnis der weiteren Kapitel erforderlich sind. Daraufhin wird im dritten Kapitel der aktuelle Stand der Forschung zusammengefasst und relevante, verwandte Webanwendungen aufgeführt, die einen Überblick über bestehende Ansätze bieten. Im vierten Kapitel liegt der Fokus auf der Konzeption und dem Entwurf von \textit{Reforge}. Hier werden die grundlegenden Designentscheidungen und die Struktur des Systems erläutert. Das darauffolgende Kapitel fünf beschreibt die konkrete Implementierung der Anwendung, wobei auf technische Details und wichtige Komponenten eingegangen wird. In Kapitel sechs werden die Ergebnisse der Arbeit präsentiert und eine Evaluation der entwickelten Lösung vorgenommen. Kapitel sieben bietet schließlich einen Ausblick auf mögliche Erweiterungen und zukünftige Verbesserungen der Anwendung. Das abschließende Kapitel acht fasst die wesentlichen Erkenntnisse dieser Arbeit zusammen und zieht ein abschließendes Fazit.
